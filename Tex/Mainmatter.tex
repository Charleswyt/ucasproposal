%---------------------------------------------------------------------------%
%->> Main content
%---------------------------------------------------------------------------%
\section{选题的背景及意义}

\subsection{选题背景}

选选题背景选题背景选题背景选题背景选题背景选题背景选题背景选题背景选题背景选题背景选题背景选题背景选题背景选题背景选题背景选题背景选题背景选题背景选题背景选题背景题背景

\subsubsection{背景}

背景背景背景背景背景背景背景背景背景背景背景背景背景背景背景背景背景背景背景背景背景背景背景背景背景背景背景背景背景背景背景背景背景背景背景背景背景背景背景背景

\subsection{选题意义}

选题意义选题意义选题意义选题意义选题意义选题意义选题意义选题意义选题意义选题意义选题意义选题意义选题意义选题意义选题意义选题意义选题意义选题意义选题意义选题意义

\subsubsection{意义}

意义意义意义意义意义意义意义意义意义意义意义意义意义意义意义意义意义意义意义意义意义意义意义意义意义意义意义意义意义意义意义意义意义意义意义意义意义意义意义意义

\section{国内外本学科领域的发展现状与趋势}

\section{课题主要研究内容、预期目标}

\section{拟采用的研究方法、技术路线、实验方案及其可行性分析}

\section{已有科研基础与所需的科研条件}

\section{研究工作计划与进度安排}

\section*{填表说明}

本表内容须真实、完整、准确。

\section*{常见使用问题}

\begin{enumerate}
    \item 模板使用说明请见 \href{https://github.com/mohuangrui/ucasthesis}{ucasthesis:中国科学院大学学位论文 LaTeX 模板}.
    \item 填表说明和模板说明不是开题报告的一部分,请删除。
    \item 开题报告样式设计导致对题目换行与不换行难以兼容,排版十分困难。推荐采用当前设置,尽量避免将精力花在这些无关紧要的细节上。
\end{enumerate}

\nocite{*}% 显示参考文献

%---------------------------------------------------------------------------%
