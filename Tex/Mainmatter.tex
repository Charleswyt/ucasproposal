%---------------------------------------------------------------------------%
%->> Main content
%---------------------------------------------------------------------------%
\section{选题的背景及意义}

\subsection{选题背景}

选选题背景选题背景选题背景选题背景选题背景选题背景选题背景选题背景选题背景选题背景选题背景选题背景选题背景选题背景选题背景选题背景选题背景选题背景选题背景选题背景题背景

\subsubsection{背景}

背景背景背景背景背景背景背景背景背景背景背景背景背景背景背景背景背景背景背景背景背景背景背景背景背景背景背景背景背景背景背景背景背景背景背景背景背景背景背景背景

\subsection{选题意义}

选题意义选题意义选题意义选题意义选题意义选题意义选题意义选题意义选题意义选题意义选题意义选题意义选题意义选题意义选题意义选题意义选题意义选题意义选题意义选题意义

\subsubsection{意义}

意义意义意义意义意义意义意义意义意义意义意义意义意义意义意义意义意义意义意义意义意义意义意义意义意义意义意义意义意义意义意义意义意义意义意义意义意义意义意义意义

\section{国内外本学科领域的发展现状与趋势}

\section{课题主要研究内容、预期目标}

\section{拟采用的研究方法、技术路线、实验方案及其可行性分析}

\section{已有科研基础与所需的科研条件}

\section{研究工作计划与进度安排}

\section*{填表说明}

本表内容须真实、完整、准确。

“学位类别”名称填写:哲学博士、教育学博士、理学博士、工学博士、农学 博士、医学博士、管理学博士,哲学硕士、经济学硕士、法学硕士、教育学 硕士、文学硕士、理学硕士、工学硕士、农学硕士、医学硕士、管理学硕士等。

“学科专业”名称填写: “二级学科”全称。

\section*{常见使用问题}

\begin{enumerate}
    \item 模板使用说明请见 \href{https://github.com/mohuangrui/ucasthesis}{中国科学院大学学位论文 LaTeX 模板 ucasthesis}.
    \item 填表说明和模板说明不是开题报告的一部分,请删除。
    \item 开题报告官方样式设计的不合理导致对题目换行与不换行难以兼容,排版十分困难。推荐采用当前设置,尽量避免将精力花在这些无关紧要的细节上。
\end{enumerate}

\nocite{*}% show all the bibliography entries

%---------------------------------------------------------------------------%
